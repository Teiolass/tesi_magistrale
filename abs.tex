\documentclass[]{marticle}
\usepackage{mstyle}

\title{\textbf{\huge Predicting and Explaining on Electronic Health Records  with Transformer-based Models: a use case}}
\date{}
\author{}


\begin{document}
\maketitle

\textbf{Candidato:} Alessio Marchetti

\textbf{Relatori:} Roberto Pellungrini, Fosca Giannotti

Rapid growth of information systems in the healthcare sector has made large amounts of relevant data
available in the last decades. One very important and rich form of this kind of data is the
Electronic Health Records (EHR), among which we can find sequences of visit descriptions taken at
irregular time distances. Visits can consist in informations on the medical conditions, symptoms,
diagnoses, medications and actions taken during the admission. 

Examples of these kinds of data are the Mimic datasets. They are collections of patient
demographics, symptoms, diagnoses, procedures and medical events gathered during the stays at the
Beth Israeli Deaconess Medical Center. Data has been processed at the Massachusetts Institute of
Technology. The last instance of the Mimic dataset is Mimic-iv, which improves on its predecessor in
several ways, one of the most important ones being the number of patients present in the data.

With large amounts of data available, a significant opportunity for the advancement of the quality
of the services offered is presented through its analysis. One important task is the prediction of
future analysis given the patient history in the form of EHR. This is a nontrivial task given the
high dimensionality of the problem and its temporal dependencies. However several approaches in the
field of deep learning have been proposed to tackle this task, mainly based on recurrent
architectures while an exploration transformer-based architectures has just started.

Good predictive models are not enough though. Models as critical as those in the healthcare sphere
have to work in tandem with professionals. To increase the trustworthiness of the models,
predictions should not only accurate but also understandable by humans. However, inner logic of
state-of the-art deep-learning models is still poorly understood. Work has been conducted in the
last years to work around this limit, an important example being doctorXAI, an agnostic explainable
model specifically built to leverage ontological information. Its core idea is to apporximate the
local behaviour of a opaque model with an intrisically explainable predictor.

DoctorXAI is composed by two parts: a black-box predictor, called doctorAI, and an explanation
pipeline. The predictor is a model based on recurrent architecture with GRU cells. The explanation
is done by training an intrinsically explainable model, a decision tree classifier, on a local
neighborhood of the data whose prediction has to be explained. The neighborhood generation is guided
by a distance on the data points which is aware of the ontology present in the ICD codes.

In this thesis we want to improve and fully understand the capabilities of doctorXAI. Our aim is to
improve the predictive accuracy of the underlying model and to provide more insight into the
explanations provided by doctorXAI, to improve its usability and effectiveness for physicians. Thus
we work in several directions.
\begin{enumerate}
\item \emph{On learning a black box model:} We substitute the underlying recurrent black-box model
with a transformer-based one, trained on an updated version of the dataset. This allows for better
performances in the predictions. We leverage on the flexibility of the positional encodings within
the transformer architecture to deal with complex data shapes, in our case sequences of subsets of a
label set.

\item \emph{On generating explanations:}  We propose a new method for generating a synthetic
neighborhood that should be aware of the underlying data distribution. This would allow for a more
thorough exploration of the local behavior of the predictive model. The process would be done by
adding noise to real data and then extracting samples over a distribution for the possible
reconstruction given by a deep learning model.

\item \emph{On explaining:} We try to explore some parts of the transformer black-box model
through the lens of the local explanation. In particular, we will notice some correlation between the
attention score given by the self-attention heads in the transformer and the feature importance
given by the tree in the local explanation pipeline. This gives interesting insights over the inner
workings of a model very similar in structure to state of the art LLMs such as GPT-3.5 or Llama but
smaller and easier to analyze.
\end{enumerate}

\end{document}

